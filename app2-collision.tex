%-------------------------------------------------------------------------------
\chapter{Impact law}
\label{app.collision}
\acresetall
%-------------------------------------------------------------------------------

Definition of impact law for the complementarity system
\cref{eq.complementarity_system} is an intricate problem \cite{Glocker2006nms}.
In the present work we do not aim for accurate modeling, and, therefore, adopt
two approximate versions of the law to simulate disturbances due to foot
touchdowns and pushes of the robot by given impulsive forces. Any disturbance
results in a discontinuous change of velocities of the system at the collision
instant. We denote the right and left limits of velocities with respect to this
instant with superscripts $(\cdot)^+$ and $(\cdot)^-$.



%%%%%%%%%%%%%%%%%%%%%%%%%%%%%%%%%%%%%%%%%%%%%%%%%%%%%%%%%%%%%%%%%%%%%%%%%%%%%%%%
%%%%%%%%%%%%%%%%%%%%%%%%%%%%%%%%%%%%%%%%%%%%%%%%%%%%%%%%%%%%%%%%%%%%%%%%%%%%%%%%
%%%%%%%%%%%%%%%%%%%%%%%%%%%%%%%%%%%%%%%%%%%%%%%%%%%%%%%%%%%%%%%%%%%%%%%%%%%%%%%%
\section{Foot touchdown}

When modeling disturbance due to a foot touchdown we assume that
%
\begin{description}
    \item[\ass{ass.nodetach}] the contacts never detach as a result of an impact;

    \item[\ass{ass.anytangentialforce}] the tangential impulsive contact forces
        are not limited due to friction;

    \item[\ass{ass.nomechanicalconstraints}] mechanical constraints of the
        system can be safely ignored, so that modeling of impacts in the joints
        can be avoided.
\end{description}
%
Under these assumptions the impact law is defined as
%
\begin{subequations}\label{eq.impact_law1}
\begin{empheq}[left=\empheqlbrace]{align}
    & \M{H}(\dq^+ - \dq^-) = \T{\Jcontact} \impulse
      ,
      \label{eq.impact_law1.dynamics}
      \\
    & \Jcontacti \dq^+ = \V{0}, & i \in \{1, ..., M\}
      ,
      \\
    & \impulseC_i^n \ge 0
      ,
      \label{eq.impact_law1.unilaterality}
\end{empheq}
\end{subequations}
%
where
%
\begin{subequations}
    \begin{align}
        & (\dcontact_1^\pm, \dots, \dcontact_M^\pm) = \Jcontact \dq^\pm
          ,
          \\
        & (\impulse_1, \dots, \impulse_M) = \impulse
          ,
          \\
        &   \begin{bmatrix}
                \dcontact^{t,\pm}_i\\
                \dcontactC^{n,\pm}_i
            \end{bmatrix}
            =
            \M[][i]{R}
            \dcontact_i^\pm,
            \enspace
            \begin{bmatrix}
                \impulse^{t}_i \\
                \impulseC^{n}_i
            \end{bmatrix}
            =
            \M[][i]{R}
            \impulse_i,
    \end{align}
\end{subequations}
%
$\impulse$ denotes impulsive contact forces, and other terms are defined in the
same way as in \cref{sec.complementarity_system}.


We obtain post-impact generalized velocities $\dq^+$ by solving an optimization
problem based on the impact law \cref{eq.impact_law1}
%
\begin{hierarchy}[hr.touchdown]
    \level $\impulseC_i^n \ge 0$\\
           $\Jcontacti \dq^+ = \V{0}$\\

    \level $\M{H}(\dq^+ - \dq^-) = \T{\Jcontact} \impulse$

    \vars{$\dq^{+}$, $\impulse$}
\end{hierarchy}
%
Exact satisfaction of all objectives is not always possible. Hence, in order to
prevent drift of the feet during simulations, we have chosen to assign higher
priority to the objective, which prevents motion of the feet.



%%%%%%%%%%%%%%%%%%%%%%%%%%%%%%%%%%%%%%%%%%%%%%%%%%%%%%%%%%%%%%%%%%%%%%%%%%%%%%%%
%%%%%%%%%%%%%%%%%%%%%%%%%%%%%%%%%%%%%%%%%%%%%%%%%%%%%%%%%%%%%%%%%%%%%%%%%%%%%%%%
%%%%%%%%%%%%%%%%%%%%%%%%%%%%%%%%%%%%%%%%%%%%%%%%%%%%%%%%%%%%%%%%%%%%%%%%%%%%%%%%
\section{Push}

We represent pushes of the robot by impulsive forces $\impulse_d$ applied at
certain points on the robot body. For this reason, the impact law
\cref{eq.impact_law1} is changed to:
%
\begin{subequations}\label{eq.impact_law2}
\begin{empheq}[left=\empheqlbrace]{align}
    & \M{H}(\dq^+ - \dq^-) = \T{\Jcontact} \impulse + \T{\M{J}_d} \impulse_d
      ,
      \\
    & \Jcontacti \dq^+ = \V{0}, & i \in \{1, ..., M\}
      ,
      \\
    & \impulseC_i^n \ge 0
      .
\end{empheq}
\end{subequations}
%
The generalized velocities $\dq^+$ after a push are found using appropriately
modified \cref{hr.touchdown}.

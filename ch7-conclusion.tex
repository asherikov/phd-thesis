%-------------------------------------------------------------------------------
\chapter{Conclusion}
\label{ch.conclusion}
\acresetall
%-------------------------------------------------------------------------------

%%%%%%%%%%%%%%%%%%%%%%%%%%%%%%%%%%%%%%%%%%%%%%%%%%%%%%%%%%%%%%%%%%%%%%%%%%%%%%%%
%%%%%%%%%%%%%%%%%%%%%%%%%%%%%%%%%%%%%%%%%%%%%%%%%%%%%%%%%%%%%%%%%%%%%%%%%%%%%%%%
%%%%%%%%%%%%%%%%%%%%%%%%%%%%%%%%%%%%%%%%%%%%%%%%%%%%%%%%%%%%%%%%%%%%%%%%%%%%%%%%
\section{Summary}

This thesis aims at understanding and developing motion control of humanoid
robots, with an ultimate goal to make motion capabilities of humans and robots
comparable. One of the key aspects of human motion is preservation of balance.
An abstract discussion in \cref{ch.balance} led us to the conclusion that
preservation of balance is equivalent to maintaining capturability, \IE, the
ability to stop. The most practical way to ensure capturability is to
anticipate motions which are constrained to end in statically balanced states.
In order to account for non-deterministic changes in the environment,
anticipation must be performed in real time, which makes anticipation of whole
body motions a particularly challenging problem. A common approach is to
sacrifice quality and completeness of anticipated motions by employing
approximate models of humanoid robots. Approximate models lack the
expressiveness of whole body models (\cref{ch.modeling}) and, hence, cannot
accurately reflect whole body tasks and constraints. We address this drawback
by the integration of an instantaneous whole body motion controller with
anticipation based on an approximate model. We call this approach \ac{MMPC},
since we mix models of different accuracy within a single predictive controller
(\cref{ch.mpc}). While this idea applies to all kinds of models, we limit the
present work to linear models, which allows us to formulate our \ac{MMPC}
controllers as \ac{PLLS} optimization problems (\cref{ch.optimization}). We
evaluate our approach in simulations using two different \ac{MMPC} controllers
(\cref{ch.simulations}). One of them is capable of adjusting steps in response
to disturbances and under the influence of whole body tasks. The second one
includes a strict prioritization of the objectives in the controller in order
to exploit additional hand contact with the environment only when it is
necessary for balance preservation or execution of a particular whole body
task. The results obtained with both controllers support the validity of our
approach. There is, of course, a number of topics for further investigation in
the considered controllers, in \ac{MMPC}, modeling of humanoid robots, and
solution of \ac{PLLS} problems. We give a brief outlook of these topics in the
following section.



%%%%%%%%%%%%%%%%%%%%%%%%%%%%%%%%%%%%%%%%%%%%%%%%%%%%%%%%%%%%%%%%%%%%%%%%%%%%%%%%
%%%%%%%%%%%%%%%%%%%%%%%%%%%%%%%%%%%%%%%%%%%%%%%%%%%%%%%%%%%%%%%%%%%%%%%%%%%%%%%%
%%%%%%%%%%%%%%%%%%%%%%%%%%%%%%%%%%%%%%%%%%%%%%%%%%%%%%%%%%%%%%%%%%%%%%%%%%%%%%%%
\section{Perspectives}

%%%%%%%%%%%%%%%%%%%%%%%%%%%%%%%%%%%%%%%%%%%%%%%%%%%%%%%%%%%%%%%%%%%%%%%%%%%%%%%%
\subsection{Mixed Model Predictive Control}

It appears that the discrepancy between the whole body motion control sampling
time and the anticipation sampling time results in periodic variations in
commands generated by \ac{MMPC} controllers. We discuss this issue in
\cref{sec.sampling_interval,sec.motion_quality,sec.force_results}, but so far
could not propose a satisfactory solution. Therefore, it is an important topic
for further research.


%%%%%%%%%%%%%%%%%%%%%%%%%%%%%%%%%%%%%%%%%%%%%%%%%%%%%%%%%%%%%%%%%%%%%%%%%%%%%%%%
\subsection{Walking using an approximate model}

The \ac{MMPC} controller for walking considered in \cref{sec.task_walk}
requires further development to make it interesting for practical applications.
Though such limitations as fixed duration and sequence of steps are unlikely to
be lifted without adoption of nonlinear approximate models, there is still room
for improvement with linear models:
%
\begin{itemize}
    \item It is possible to enable walking with varying \ac{CoM} height using
        the recent proposal from \cite{Brasseur2015humanoids}.

    \item We can abandon the point-mass model and employ a model including
        angular momentum. Incorporation of angular momentum in the
        capturability constraint may help to avoid situations when the
        currently used capturability constraint is not sufficient for
        preservation of balance (see \cref{sec.walk_capturability}).
        Furthermore, angular momentum allows to account for motions of heavy
        end-effectors, \EG, legs, in the anticipated motions (see
        \cref{sec.approx_models_limitations}).

    \item So far we used simple cubic polynomials for generation of foot
        trajectories. Instead, we can adopt a triple integrator for this
        purpose, in the same way as we do for the \ac{CoM} motion in
        \nameref{model.CPPMJ} model. This modification introduces additional
        freedom in trajectory generation and allows for variation of the step
        height or simple obstacle avoidance.

    \item In this work we did not realize rotations of the robot and its feet,
        which is a significant drawback. In general, rotations result in
        nonlinear constraints on the \ac{CoP} positions and positions of the
        feet. We can address the first problem by shrinking the \ac{CoP}
        constraints as proposed in \cref{sec.surface_contacts}. The second
        problem can be avoided by reducing the length of the preview horizon
        from $2$ to $1$ step, which we have already demonstrated to be possible
        in \cref{sec.walk_performance}.
\end{itemize}
%
Last, but not least is the experimental evaluation of the controller on a real
robot.


%%%%%%%%%%%%%%%%%%%%%%%%%%%%%%%%%%%%%%%%%%%%%%%%%%%%%%%%%%%%%%%%%%%%%%%%%%%%%%%%
\subsection{Prioritization in contact force distribution}

Further development of the idea of prioritization in contact force distribution
for partial contact planning as explained in \cref{sec.force_conclusion} is
appealing, but may require significant improvements in numerical tools to make
it applicable in practice.


%%%%%%%%%%%%%%%%%%%%%%%%%%%%%%%%%%%%%%%%%%%%%%%%%%%%%%%%%%%%%%%%%%%%%%%%%%%%%%%%
\subsection{Solvers for optimization problems with prioritization}

We see two primary directions for improvement of the existing solvers of
\acf{PLLS} problems:
%
\begin{itemize}
    \item We would like the solvers to be able to exploit general sparsity
        patterns in the objectives for improvement of performance and not only
        simple bounds. Ideally, we would like to avoid any manual variable
        elimination steps before solution of a \ac{PLLS} problem.

    \item The current mechanisms for coping with issues near singularities
        appear to be insufficient and require further development (see
        \cref{sec.force_conclusion}).
\end{itemize}

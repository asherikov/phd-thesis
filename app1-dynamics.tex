%-------------------------------------------------------------------------------
\chapter{Dynamics of a multibody system}
\label{app.dynamics}
\acresetall
%-------------------------------------------------------------------------------

In this appendix we derive equation of dynamics of a floating base multibody
system with unilateral constraints. The presentation is based on
\cite{Wieber2006fastmotions}, and likewise aims at exposing the structure of
this equation, rather than at fast computation of its components. In addition
to that, we demonstrate how momenta of the whole system are computed.



%%%%%%%%%%%%%%%%%%%%%%%%%%%%%%%%%%%%%%%%%%%%%%%%%%%%%%%%%%%%%%%%%%%%%%%%%%%%%%%%
%%%%%%%%%%%%%%%%%%%%%%%%%%%%%%%%%%%%%%%%%%%%%%%%%%%%%%%%%%%%%%%%%%%%%%%%%%%%%%%%
%%%%%%%%%%%%%%%%%%%%%%%%%%%%%%%%%%%%%%%%%%%%%%%%%%%%%%%%%%%%%%%%%%%%%%%%%%%%%%%%
\section{Preliminaries}

%%%%%%%%%%%%%%%%%%%%%%%%%%%%%%%%%%%%%%%%%%%%%%%%%%%%%%%%%%%%%%%%%%%%%%%%%%%%%%%%
\subsection{Notation}

We consider a system of $n + 1$ interconnected rigid bodies. We associate the
following variables with each body $k \in \{1, ..., n+1\}$
%
\begin{longtable}[l]{@{\extracolsep{0pt}}l @{\extracolsep{3pt}}l p{9.5cm}}
    $m_k$                                           & $\in \RR_{>0}$                 & mass of $k$-th body\\
    $\V{c}_k$                                       & $\in \RR^3$                   & position of \ac{CoM} of $k$-th body in the global frame\\
    $\ddotV{x}_k = (\ddotV{c}_k, \dotV{\omega}_k)$  & $\in \RR^6$                   & constrained spatial acceleration of $k$-th body in frame $\FRAME{k}$\\
    $\ddotV{x}_{k,u}$                               & $\in \RR^6$                   & unconstrained spatial acceleration of $k$-th body in frame $\FRAME{k}$\\
    $\INERTIA_k$                                    & $\in \RR^{3 \CROSS 3}$        & inertia matrix of $k$-th body in frame $\FRAME{k}$\\
    $\M{H}_k = \begin{bmatrix}
                    m_k \V{I}_3 & \V{0}      \\
                    \V{0}       & \INERTIA_k
               \end{bmatrix}$                       & $\in \RR^{6 \CROSS 6}$        & spatial inertia matrix of $k$-th body in frame $\FRAME{k}$\\
    $(\V{f}_k, \moment_k)$                          & $\in \RR^{6 \CROSS 6}$        & wrench acting on $k$-th body in frame $\FRAME{k}$\\
    $\M{J}_k = (\M{J}_{\TRAN,k}, \M{J}_{\ROT,k})$   & $\in \RR^{6 \CROSS (n+6)}$      & Jacobian of $k$-th body\\
\end{longtable}
%
\noindent and fix frame $\FRAME{k}$ to the \ac{CoM} of $k$-th body. All frames
$\FRAME{k}$, as well as frame $\FRAME{r}$ fixed to the base have the same
orientation as the global frame. Also, we reuse some of the variables defined
in \cref{sec.complementarity_system}
%
\begin{longtable}[l]{@{\extracolsep{0pt}}l @{\extracolsep{3pt}}l p{9.5cm}}
    $\q = (\qn, \V{r}, \V{\EULER})$         & $\in \RR^{n+6}$             & vector of generalized coordinates\\
    $\M{H}(\q)$                             & $\in \RR^{(n+6) \CROSS (n+6)}$    & inertia matrix of the whole system\\
    $\V{h}(\q, \dq)$                        & $\in \RR^{n+6}$               & vector of Coriolis and centrifugal terms of the whole system\\
    $\Jcom$                                 & $\in \RR^{3 \CROSS (n+6)}$      & Jacobian of the \ac{CoM}\\
    $\Itorques$                             & $\in \RR^{(n+6) \CROSS n}$      & torque selection matrix\\
    $\V{g}$                                 & $\in \RR^3$                   & vector of gravitational acceleration\\
    $m$                                     & $\in \RR_{>0}$                 & total mass of the system\\
\end{longtable}
%


%%%%%%%%%%%%%%%%%%%%%%%%%%%%%%%%%%%%%%%%%%%%%%%%%%%%%%%%%%%%%%%%%%%%%%%%%%%%%%%%
\subsection{Structure of Jacobians}\label{sec.jacobians}

Vector of generalized coordinates $\V{q}$ contains three parts corresponding to
joint angles, position, and orientation of the base. Consequently, the Jacobians,
which map generalized velocities to the twist of the $k$-th body
%
\begin{equation}
    \begin{bmatrix}
        \dotV{c}_k \\
        \V{\omega}_k
    \end{bmatrix}
    =
    \begin{bmatrix}
        \M{J}_{\TRAN,k}\\
        \M{J}_{\ROT,k}\\
    \end{bmatrix}
    \dq
    =
    \M{J}_k
    \dq
    ,
\end{equation}
%
have corresponding structure:
%
\begin{equation}
    \M{J}_k
    =
    \begin{bmatrix}
        \M{J}_{\TRAN,k}\\
        \M{J}_{\ROT,k}\\
    \end{bmatrix}
    =
    \begin{bmatrix}
        \M{J}_{\TRAN,k,1}  &  \M{J}_{\TRAN,k,2}  &  \M{J}_{\TRAN,k,3}\\
        \M{J}_{\ROT,k,1}  &  \M{J}_{\ROT,k,2}  &  \M{J}_{\ROT,k,3}\\
    \end{bmatrix}
    =
    \begin{bmatrix}
        \M{J}_{\TRAN,k,1}  &  \M{I}_3  &  - \CROSS[(\V{c}_k - \V{r})] \\
        \M{J}_{\ROT,k,1}  &  \M{0}_{3,3}  &  \M{I}_3\\
    \end{bmatrix}
\end{equation}
%


In situations when orientation of the base is represented by Euler angles
$\V{\EULER} \in \RR^3$ and its angular velocity and acceleration are replaced
with $\dotV{\EULER}$ and $\ddotV{\EULER}$, it is necessary to introduce matrix
$\Teuler$, which transforms derivatives of the Euler angles to angular
velocities:
%
\begin{equation}
    \M{J}_k
    =
    \begin{bmatrix}
        \M{J}_{\TRAN,k,1}  &  \M{I}_3  &  - \CROSS[(\V{c}_k - \V{r})] \Teuler\\
        \M{J}_{\ROT,k,1}  &  \M{0}_{3,3}  &  \Teuler\\
    \end{bmatrix}
    =
    \begin{bmatrix}
        \M{J}_{\TRAN,k,1}  &  \M{I}_3  &  \T{(\CROSS[(\V{c}_k - \V{r})])} \Teuler\\
        \M{J}_{\ROT,k,1}  &  \M{0}_{3,3}  &  \Teuler\\
    \end{bmatrix}
\end{equation}
%
Henceforth, we use this version of the Jacobian, since we represent
orientation, angular velocity and acceleration of the base using Euler angles
in our simulations.


%%%%%%%%%%%%%%%%%%%%%%%%%%%%%%%%%%%%%%%%%%%%%%%%%%%%%%%%%%%%%%%%%%%%%%%%%%%%%%%%
%%%%%%%%%%%%%%%%%%%%%%%%%%%%%%%%%%%%%%%%%%%%%%%%%%%%%%%%%%%%%%%%%%%%%%%%%%%%%%%%
%%%%%%%%%%%%%%%%%%%%%%%%%%%%%%%%%%%%%%%%%%%%%%%%%%%%%%%%%%%%%%%%%%%%%%%%%%%%%%%%
\section{Gauss' principle}

The Gauss' principle states that constrained motions of rigid bodies in a
system are as close as possible to the unconstrained motions in least-squares
sense \cite{Wieber2006fastmotions, Moreau1966siamjc}. Based on this principle
we define the following optimization problem:
%
\begin{equation}
    \begin{aligned}
            \MINIMIZE{\ddotV{x}_1, \dots, \ddotV{x}_n, \ddot{q}}
                        & \frac{1}{2} \sum_{k=1}^{n+1}
                            \T{(\ddotV{x}_k - \ddotV{x}_{k,u})}
                            \M{H}_k
                            (\ddotV{x}_k - \ddotV{x}_{k,u}) \\
            \SUBJECTTO  & \ddotV{x}_k =
                            \begin{bmatrix}
                                \ddotV{c}_k\\
                                \dotV{\omega}_k\\
                            \end{bmatrix}
                            =
                            \begin{bmatrix}
                                \M{J}_{\TRAN,k}\\
                                \M{J}_{\ROT,k}\\
                            \end{bmatrix}
                            \ddotV{q}
                            +
                            \begin{bmatrix}
                                \dotM{J}_{\TRAN,k}\\
                                \dotM{J}_{\ROT,k}\\
                            \end{bmatrix}
                            \dotV{q}
                            =
                            \M{J}_k \ddotV{q} + \dotM{J}_k \dotV{q} \\
                        & \M{H}_k \ddotV{x}_{k,u} =
                            \begin{bmatrix}
                                \V{f}_k\\
                                \moment_k - \V{\omega}_k \CROSS \INERTIA_k \V{\omega}_k \\
                            \end{bmatrix}\\
                        & \ddotV{c}_j =
                            \M{J}_{\TRAN,j}
                            \ddotV{q}
                            +
                            \dotM{J}_{\TRAN,j}
                            \dotV{q}
                            \ge
                            \V{b}_j,
    \end{aligned}
\end{equation}
%
where inequality constraint on $\ddotV{c}_j$ with $j \in \{1, ..., n+1\}$ is
added for illustration. Handling of this constraint is easily generalized to
arbitrary inequalities, such as contact constraints in
\cref{sec.whole_body_model}.


First, we expand multiplication in the objective function and eliminate
constant terms to obtain:
%
\begin{equation}
    \begin{aligned}
            \MINIMIZE{\ddotV{x}_1, \dots, \ddotV{x}_n, \ddot{q}}
                        & \frac{1}{2} \sum_{k=1}^{n+1}
                            \left(
                                \T{\ddotV{x}_k}  \M{H}_k  \ddotV{x}_k
                                -
                                2 \T{\ddotV{x}_k}  \M{H}_k  \ddotV{x}_{k,u}
                            \right)\\
            \SUBJECTTO  & \ddotV{x}_k =
                            \M{J}_k \ddotV{q} + \dotM{J}_k \dotV{q} \\
                        & \M{H}_k \ddotV{x}_{k,u} =
                            \begin{bmatrix}
                                \V{f}_k\\
                                \moment_k - \V{\omega}_k \CROSS \INERTIA_k \V{\omega}_k \\
                            \end{bmatrix}\\
                        &   \M{J}_{\TRAN,j}
                            \ddotV{q}
                            +
                            \dotM{J}_{\TRAN,j}
                            \dotV{q}
                            \ge
                            \V{b}_j.
    \end{aligned}
\end{equation}
%
Then we eliminate all equality constraints
%
\begin{equation}
    \begin{aligned}
            \MINIMIZE{\ddot{q}}
                        & \frac{1}{2} \sum_{k=1}^{n+1}
                            \Bigg(
                            \T{(\M{J}_k \ddotV{q})}  \M{H}_k  \M{J}_k \ddotV{q}
                            +
                            2 \T{(\M{J}_k \ddotV{q})}  \M{H}_k  \dotM{J}_k \dotV{q}
                            +
                            \T{(\dotM{J}_k \dotV{q})}  \M{H}_k  \dotM{J}_k \dotV{q}\\
                        &
                            -
                            2 \T{(\M{J}_k \ddotV{q} + \dotM{J}_k \dotV{q})}
                            \begin{bmatrix}
                                \V{f}_k\\
                                \moment_k - \V{\omega}_k \CROSS \INERTIA_k \V{\omega}_k \\
                            \end{bmatrix} \Bigg)\\
            \SUBJECTTO  &   \M{J}_{\TRAN,j}
                            \ddotV{q}
                            +
                            \dotM{J}_{\TRAN,j}
                            \dotV{q}
                            \ge
                            \V{b}_j,
    \end{aligned}
\end{equation}
%
and drop the constant terms
%
\begin{equation}
    \begin{aligned}
            \MINIMIZE{\ddot{q}}
                        & \sum_{k=1}^{n+1}
                            \left(
                            \frac{1}{2} \T{(\M{J}_k \ddotV{q})}  \M{H}_k  \M{J}_k \ddotV{q}
                            +
                            \T{(\M{J}_k \ddotV{q})}  \M{H}_k  \dotM{J}_k \dotV{q}
                            -
                            \T{(\M{J}_k \ddotV{q})}
                            \begin{bmatrix}
                                \V{f}_k\\
                                \moment_k - \V{\omega}_k \CROSS \INERTIA_k \V{\omega}_k \\
                            \end{bmatrix}
                            \right)\\
            \SUBJECTTO  &   \M{J}_{\TRAN,j}
                            \ddotV{q}
                            +
                            \dotM{J}_{\TRAN,j}
                            \dotV{q}
                            \ge
                            \V{b}
    \end{aligned}
\end{equation}
%


\ac{KKT} optimality conditions for the considered problem are stated as a
complementarity system \cite{Nocedal2006numopt}
\begin{subequations}\label{eq.kkt_dynamics}
\begin{empheq}[left=\empheqlbrace]{align}
    &   \sum_{k=1}^{n+1}
        \left(
            \T{\M{J}_k}  \M{H}_k  \M{J}_k \ddotV{q}
            +
            \T{\M{J}_k}  \M{H}_k  \dotM{J}_k \dotV{q}
            -
            \T{\M{J}_k}
            \begin{bmatrix}
                \V{f}_k\\
                \moment_k - \V{\omega}_k \CROSS \INERTIA_k \V{\omega}_k \\
            \end{bmatrix}
        \right)
        -
        \T{\M{J}_{\TRAN,j}}
        \V{\Lambda}
        = 0, \label{eq.kkt_dynamics.dynamics}\\
    &   \M{J}_{\TRAN,j}
        \ddotV{q}
        +
        \dotM{J}_{\TRAN,j}
        \dotV{q}
        \ge
        \V{b},\\
    &   \V{\Lambda} \ge \V{0},\\
    &   \T{\V{\Lambda}}
        \left(
            \M{J}_{\TRAN,j}
            \ddotV{q}
            +
            \dotM{J}_{\TRAN,j}
            \dotV{q}
            -
            \V{b}
        \right)
        =
        \V{0},
\end{empheq}
\end{subequations}
where $\V{\Lambda}$ is the vector of Lagrange multipliers.
System~\cref{eq.kkt_dynamics} corresponds to the complementarity system
presented in \cref{sec.whole_body_model}: the first line is the equation of
dynamics of the whole system and $\V{\Lambda}$ correspond to the contact
forces.


We further modify \cref{eq.kkt_dynamics.dynamics} by substituting
$\V{\omega}_k = \M{J}_{\ROT,k} \dotV{q}$ and moving all forces to the right
side:
%
\begin{equation}\label{eq.lagrangian_dynamics}
    \underbrace{
        \sum_{k=1}^{n+1}
        \left(
            \T{\M{J}_k}  \M{H}_k  \M{J}_k
        \right)
    }_{\M{H}}
    \ddotV{q}
    +
    \underbrace{
        \sum_{k=1}^{n+1}
        \left(
            \T{\M{J}_k}  \M{H}_k  \dotM{J}_k \dotV{q}\\
            +
            \T{\M{J}_{\ROT,k}}
            ((\M{J}_{\ROT,k} \dotV{q}) \CROSS \INERTIA_k \M{J}_{\ROT,k} \dotV{q})
        \right)
    }_{\V{h}}
    =
    \sum_{k=1}^{n+1}
    \T{\M{J}_k}
    \begin{bmatrix}
        \V{f}_k\\
        \moment_k\\
    \end{bmatrix}
    +
    \T{\M{J}_{\TRAN,j}}
    \V{\Lambda}.
\end{equation}
%


%%%%%%%%%%%%%%%%%%%%%%%%%%%%%%%%%%%%%%%%%%%%%%%%%%%%%%%%%%%%%%%%%%%%%%%%%%%%%%%%
%%%%%%%%%%%%%%%%%%%%%%%%%%%%%%%%%%%%%%%%%%%%%%%%%%%%%%%%%%%%%%%%%%%%%%%%%%%%%%%%
%%%%%%%%%%%%%%%%%%%%%%%%%%%%%%%%%%%%%%%%%%%%%%%%%%%%%%%%%%%%%%%%%%%%%%%%%%%%%%%%
\section{Structure of the equation of dynamics}

In this section we employ our knowledge of the structure of Jacobians to
explore the structure of components of Equation~\cref{eq.lagrangian_dynamics}.


%%%%%%%%%%%%%%%%%%%%%%%%%%%%%%%%%%%%%%%%%%%%%%%%%%%%%%%%%%%%%%%%%%%%%%%%%%%%%%%%
\subsection{Inertia matrix}

Expansion of the Jacobians exposes the structure of inertia matrix as shown
below
%
\begin{subequations}
\begin{align}
    \M{H}
    & =
    \sum_{k=1}^{n+1}
    \T{\M{J}_k}  \M{H}_k  \M{J}_k \\
    & =
    \sum_{k=1}^{n+1}
    m_k \T{\M{J}_{\TRAN,k}}  \M{J}_{\TRAN,k}
    +
    \sum_{k=1}^{n+1}
    \T{\M{J}_{\ROT,k}}  \INERTIA_k  \M{J}_{\ROT,k} \\
    & =
    \sum_{k=1}^{n+1}
    \left(
        m_k
        \begin{bmatrix}
            \T{\M{J}_{\TRAN,k,1}}  \M{J}_{\TRAN,k}\\
            \T{\M{J}_{\TRAN,k,2}}  \M{J}_{\TRAN,k}\\
            \T{\M{J}_{\TRAN,k,3}}  \M{J}_{\TRAN,k}\\
        \end{bmatrix}
        +
        \begin{bmatrix}
            \T{\M{J}_{\ROT,k,1}} \INERTIA_k \M{J}_{\ROT,k}\\
            \T{\M{J}_{\ROT,k,2}} \INERTIA_k \M{J}_{\ROT,k}\\
            \T{\M{J}_{\ROT,k,3}} \INERTIA_k \M{J}_{\ROT,k}\\
        \end{bmatrix}
    \right)\\
    \begin{split}
    &
    =
    \sum_{k=1}^{n+1}
    \Bigg(
        m_k
        \begin{bmatrix}
            \T{\M{J}_{\TRAN,k,1}}  \M{J}_{\TRAN,k,1}                    & \T{\M{J}_{\TRAN,k,1}}                 & - \T{\M{J}_{\TRAN,k,1}}  \CROSS[(\V{c}_k - \V{r})] \Teuler\\
            \M{J}_{\TRAN,k,1}                                           & \V{I}_3                               & - \CROSS[(\V{c}_k - \V{r})] \Teuler\\
            \T{\Teuler} \CROSS[(\V{c}_k - \V{r})]  \M{J}_{\TRAN,k,1}  & \T{\Teuler} \CROSS[(\V{c}_k - \V{r})] & - \T{\Teuler} \CROSS[(\V{c}_k - \V{r})] \CROSS[(\V{c}_k - \V{r})] \Teuler\\
        \end{bmatrix}\\
        & \quad \quad \quad \quad +
        \begin{bmatrix}
            \T{\M{J}_{\ROT,k,1}} \INERTIA_k \M{J}_{\ROT,k,1}    & \M{0}_{n,3} & \T{\M{J}_{\ROT,k,1}} \INERTIA_k \Teuler\\
            \M{0}_{3,n}                                         & \M{0}_{3,3} & \M{0}_{3,3}\\
            \T{\Teuler} \INERTIA_k \M{J}_{\ROT,k,1}             & \M{0}_{3,3} & \T{\Teuler} \INERTIA_k \Teuler\\
        \end{bmatrix}
    \Bigg)
    \end{split}
    \label{eq.inertia_structure.long_line}
    \\
    & =
    \sum_{k=1}^{n+1}
    \left(
        m_k
        \begin{bmatrix}
            \T{\M{J}_{\TRAN,k,1}}  \M{J}_{\TRAN,k}\\
            \M{J}_{\TRAN,k}                     \\
            \T{\Teuler} \CROSS[(\V{c}_k - \V{r})] \M{J}_{\TRAN,k}\\
        \end{bmatrix}
        +
        \begin{bmatrix}
            \T{\M{J}_{\ROT,k,1}} \INERTIA_k \M{J}_{\ROT,k}\\
            \M{0}_{n+6,3}                                              \\
            \T{\Teuler} \INERTIA_k \M{J}_{\ROT,k}\\
        \end{bmatrix}
    \right)
    =
    \begin{bmatrix}
        \M{H}_{1} \\
        \M{H}_{2} \\
        \M{H}_{3} \\
    \end{bmatrix}
\end{align}
\end{subequations}
%

It is worth noting that the structure of the inertia matrix simplifies when the
base $\V{r}$ is chosen to coincide with the \ac{CoM} $\V{c}$. In this case some
of the components of $\M{H}$ are equal to zero due to
%
\begin{equation}
    \sum_{k=1}^{n+1}
        m_k \CROSS[(\V{c}_k - \V{r})]
    =
    \sum_{k=1}^{n+1}
        \CROSS[(m_k \V{c}_k - m_k \V{c})]
    =
    \CROSS[\left(m \V{c} - \sum_{k=1}^{n+1} (m_k \V{c})\right)]
    =
    \CROSS[(\M{0}_{3,1})]
    =
    \M{0}_{3,3}.
\end{equation}
%


%%%%%%%%%%%%%%%%%%%%%%%%%%%%%%%%%%%%%%%%%%%%%%%%%%%%%%%%%%%%%%%%%%%%%%%%%%%%%%%%
\subsection{Nonlinear term}
%
\begin{equation}
\begin{aligned}
    &\V{h}
     =
    \sum_{k=1}^{n+1}
    \left(
        \M{J}_k^T  \M{H}_k  \dotM{J}_k \dotV{q}
        +
        \M{J}_{\ROT,k}^T
        ((\M{J}_{\ROT,k} \dotV{q}) \CROSS \INERTIA_k \M{J}_{\ROT,k} \dotV{q})
    \right)\\
    & =
    \sum_{k=1}^{n+1}
    \left(
        m_k
        \begin{bmatrix}
            \M{J}_{\TRAN,k,1}^T  \dotM{J}_{\TRAN,k} \dotV{q}\\
            \dotM{J}_{\TRAN,k} \dotV{q}                     \\
            \T{\Teuler} \CROSS[(\V{c}_k - \V{r})] \dotM{J}_{\TRAN,k} \dotV{q}\\
        \end{bmatrix}
        +
        \begin{bmatrix}
            \M{J}_{\ROT,k,1}^T \INERTIA_k \dotM{J}_{\ROT,k} \dotV{q} + \M{J}_{\ROT,k,1}^T ((\M{J}_{\ROT,k} \dotV{q}) \CROSS \INERTIA_k \M{J}_{\ROT,k} \dotV{q})\\
            \M{0}_{3,1} \\
            \T{\Teuler} \INERTIA_k \dotM{J}_{\ROT,k} \dotV{q}        + \T{\Teuler} ((\M{J}_{\ROT,k} \dotV{q}) \CROSS \INERTIA_k \M{J}_{\ROT,k} \dotV{q})\\
        \end{bmatrix}
    \right)
    \\
    & =
    \begin{bmatrix}
        \M{h}_{1} \\
        \M{h}_{2} \\
        \M{h}_{3} \\
    \end{bmatrix}
\end{aligned}
\end{equation}
%


%%%%%%%%%%%%%%%%%%%%%%%%%%%%%%%%%%%%%%%%%%%%%%%%%%%%%%%%%%%%%%%%%%%%%%%%%%%%%%%%
\subsection{Forces}

Assuming that the wrenches $(\V{f}_k, \moment_k)$ are the result of gravity
field and joint torques we obtain
%
\begin{equation}
    \sum_{k=1}^{n+1}
    \T{\M{J}_k}
    \begin{bmatrix}
        \V{f}_k\\
        \moment_k\\
    \end{bmatrix}
    +
    \T{\M{J}_{\TRAN,j}}
    \V{\Lambda}
    =
    \sum_{k=1}^{n+1}
    \left(
        m_k
        \begin{bmatrix}
            \M{J}_{\TRAN,k,1}^T\\
            \M{I}_3                     \\
            \T{\Teuler} \CROSS[(\V{c}_k - \V{r})]\\
        \end{bmatrix}
    \right)
    \V{g}
    +
    \Itorques
    \torques
    +
    \begin{bmatrix}
        \M{J}_{\TRAN,j,1}^T\\
        \M{I}_3                     \\
        \T{\Teuler} \CROSS[(\V{c}_j - \V{r})]\\
    \end{bmatrix}
    \V{\Lambda}.
\end{equation}
%



%%%%%%%%%%%%%%%%%%%%%%%%%%%%%%%%%%%%%%%%%%%%%%%%%%%%%%%%%%%%%%%%%%%%%%%%%%%%%%%%
%%%%%%%%%%%%%%%%%%%%%%%%%%%%%%%%%%%%%%%%%%%%%%%%%%%%%%%%%%%%%%%%%%%%%%%%%%%%%%%%
%%%%%%%%%%%%%%%%%%%%%%%%%%%%%%%%%%%%%%%%%%%%%%%%%%%%%%%%%%%%%%%%%%%%%%%%%%%%%%%%
\section{Momenta of the system}\label{sec.momenta_structure}

Consider the lower $6$ lines of the equation of dynamics:
%
\begin{equation}
\begin{aligned}
    &
    \begin{bmatrix}
        \M{I}_3   &   \M{0}_{3,3} \\
        \M{0}_{3,3}   &   \T{\Teuler}\\
    \end{bmatrix}
    \Bigg(
        \sum_{k=1}^{n+1}
        \left(
            m_k
            \begin{bmatrix}
                \M{J}_{\TRAN,k}                     \\
                \CROSS[(\V{c}_k - \V{r})] \M{J}_{\TRAN,k}\\
            \end{bmatrix}
            +
            \begin{bmatrix}
                \M{0}_{3,n+6}                                              \\
                \INERTIA_k \M{J}_{\ROT,k}\\
            \end{bmatrix}
        \right)
        \ddotV{q}
        \\
        &+
        \sum_{k=1}^{n+1}
        \left(
            m_k
            \begin{bmatrix}
                \dotM{J}_{\TRAN,k} \dotV{q}                     \\
                \CROSS[(\V{c}_k - \V{r})] \dotM{J}_{\TRAN,k} \dotV{q}\\
            \end{bmatrix}
            +
            \begin{bmatrix}
                \M{0}_{3,1}                                              \\
                \INERTIA_k \dotM{J}_{\ROT,k} \dotV{q}\\
            \end{bmatrix}
            +
            \begin{bmatrix}
                \M{0}_{3,1}                                          \\
                ((\M{J}_{\ROT,k} \dotV{q}) \CROSS \INERTIA_k \M{J}_{\ROT,k} \dotV{q})\\
            \end{bmatrix}
        \right)
    \Bigg)
    \\
    & =
    \begin{bmatrix}
        \M{I}_3   &   \M{0}_{3,3} \\
        \M{0}_{3,3}   &   \T{\Teuler}\\
    \end{bmatrix}
    \Bigg(
        \sum_{k=1}^{n+1}
        \left(
            m_k
            \begin{bmatrix}
                \M{I}_3                     \\
                \CROSS[(\V{c}_k - \V{r})]\\
            \end{bmatrix}
        \right)
        \V{g}
        +
        \begin{bmatrix}
            \M{I}_3                     \\
            \CROSS[(\V{c}_j - \V{r})]\\
        \end{bmatrix}
        \V{\Lambda}
    \Bigg)
    .
\end{aligned}
\end{equation}
%
%
This is the Newton-Euler equation of the whole system multiplied by the
constant matrix $\diag{}{\M{I}_3, \T{\Teuler}}$ \cite{Wieber2006fastmotions}.
The right hand side of the equation gives the total wrench acting at the
reference point $\V{r}$ of the system, hence
%
\begin{equation}
\begin{aligned}
    &\sum_{k=1}^{n+1}
    \left(
        m_k
        \begin{bmatrix}
            \M{J}_{\TRAN,k}                     \\
            \CROSS[(\V{c}_k - \V{r})] \M{J}_{\TRAN,k}\\
        \end{bmatrix}
        +
        \begin{bmatrix}
            \M{0}_{3,n+6}                                              \\
            \INERTIA_k \M{J}_{\ROT,k}\\
        \end{bmatrix}
    \right)
    \ddotV{q}
    \\
    &+
    \sum_{k=1}^{n+1}
    \left(
        m_k
        \begin{bmatrix}
            \dotM{J}_{\TRAN,k} \dotV{q}                     \\
            \CROSS[(\V{c}_k - \V{r})] \dotM{J}_{\TRAN,k} \dotV{q}\\
        \end{bmatrix}
        +
        \begin{bmatrix}
            \M{0}_{3,1}                                              \\
            \INERTIA_k \dotM{J}_{\ROT,k} \dotV{q}\\
        \end{bmatrix}
        +
        \begin{bmatrix}
            \M{0}_{3,1}                                              \\
            ((\M{J}_{\ROT,k} \dotV{q}) \CROSS \INERTIA_k \M{J}_{\ROT,k} \dotV{q})\\
        \end{bmatrix}
    \right)
    \\
    &=
    \begin{bmatrix}
        \M{I}_3   &   \M{0}_{3,3} \\
        \M{0}_{3,3}   &   \TI{\Teuler}\\
    \end{bmatrix}
    \left(
        \begin{bmatrix}
            \M{H}_2\\
            \M{H}_3\\
        \end{bmatrix}
        \ddotV{q}
        +
        \begin{bmatrix}
            \M{h}_2\\
            \M{h}_3\\
        \end{bmatrix}
    \right)
    =
    \begin{bmatrix}
        \dotV[r]{\LM}\\
        \dotV[r]{\AM}
    \end{bmatrix},
\end{aligned}
\end{equation}
%
where $\dotV[r]{\LM}$ and $\dotV[r]{\AM}$ are the rates of linear and angular
momenta of the system expressed in frame $\FRAME{r}$. Integration of this
equation allows to compute the momenta as
%
\begin{equation}
    \sum_{k=1}^{n+1}
    \left(
        m_k
        \begin{bmatrix}
            \M{J}_{\TRAN,k}                     \\
            \CROSS[(\V{c}_k - \V{r})] \M{J}_{\TRAN,k}\\
        \end{bmatrix}
        +
        \begin{bmatrix}
            \M{0}_{3,n+6}                                              \\
            \INERTIA_k \M{J}_{\ROT,k}\\
        \end{bmatrix}
    \right)
    \dotV{q}
    =
    \begin{bmatrix}
        \M{I}_3   &   \M{0}_{3,3} \\
        \M{0}_{3,3}   &   \TI{\Teuler}\\
    \end{bmatrix}
    \begin{bmatrix}
        \M{H}_2\\
        \M{H}_3\\
    \end{bmatrix}
    \dotV{q}
    =
    \begin{bmatrix}
        \V[r]{\LM}\\
        \V[r]{\AM}
    \end{bmatrix}.
\end{equation}
%
Then, assuming that frame $\FRAME{r}$ has the same orientation as the global
frame
%
\begin{equation}
    \V[r]{\LM} = \V{\LM} = m \dotV{c} =
    \sum_{k=1}^{n+1}
    \left(
        m_k \M{J}_{\TRAN,k}
    \right)
    \dotV{q},
    \quad
    \dotV{c}
    =
    \left(
        \frac{1}{m}
        \sum_{k=1}^{n+1}
            m_k \M{J}_{\TRAN,k}
    \right)
    \dotV{q}
    =
    \Jcom
    \dotV{q},
\end{equation}
%
where $\Jcom$ is Jacobian of the \ac{CoM} in accordance with
\cite{Sugihara2002iros, Espiau1998report}.


Often, it is more convenient to work with momenta $\V{\LM}$ and $\V{\AM}$
expressed in a frame $\FRAME{c}$ fixed to the \ac{CoM} of the system
\cite{Orin2013auro}. If this frame has the same orientation as frame
$\FRAME{r}$ transformation of the linear momentum is not needed $\V[c]{\LM} =
\V[r]{\LM}$. In order to transform angular momentum we use $\V[r]{\AM} =
\V[c]{\AM} + m \V[r]{c} \CROSS \dotV[r]{c}$, where $\V[r]{c} = \V{c} - \V{r}$
is the vector from the base to the \ac{CoM}:
%
\begin{equation}\label{eq.angular_momentum_com}
    \V[c]{\AM}
    =
    \V[r]{\AM} - m \V[r]{c} \CROSS \dotV[r]{c}
    =
    \V[r]{\AM}
    -
    m
    \V[r]{c}
    \CROSS
    (
        \Jcom
        \dotV{q}
    )
    =
    \V[r]{\AM}
    -
    \V[r]{c}
    \CROSS
    \V[r]{\LM}.
\end{equation}
%
Rate of the angular momentum $\dotV[c]{\AM}$ is computed by differentiating
\cref{eq.angular_momentum_com}:
%
\begin{equation}
    \dotV[c]{\AM}
    =
    \dotV[r]{\AM}
    -
    \dotV{c}
    \CROSS
    \V[r]{\LM}
    -
    \V[r]{c}
    \CROSS
    \dotV[r]{\LM}.
\end{equation}
%
Note, that computation of momenta with respect to the \ac{CoM} as suggested in
\cite{Orin2013auro} is equivalent to computation of the equation of dynamics
with $\V{r} = \V{c}$, and it is unnecessary if equation of dynamics with $\V{r}
\neq \V{c}$ is already constructed.

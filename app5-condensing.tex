%-------------------------------------------------------------------------------
\chapter{Condensing of a Model Predictive Control problem}
\label{app.condensing}
\acresetall
%-------------------------------------------------------------------------------

In this appendix we illustrate the condensing procedure employed in the
\ac{MPC} field in order to eliminate some of the decision variables
\cite{Bock1984ifac}.

Consider a linear model
%
\begin{equation}
\begin{aligned}
    \V{x}_{k+1} =& \M{A}_k \V{x}_k + \M{B}_k \V{u}_k, \quad k \in \{0, ..., N-1\}\\
\end{aligned}
\end{equation}
%
where $\V{x}_k$ and $\V{u}_k$ are $k$-th state and control vectors
respectively, and $N$ is the length of preview horizon.


Condensing amounts to finding such matrices $\M{U}_{x}$ and $\M{U}_{u}$ that
%
\begin{equation}
\begin{aligned}
    \hatV{x} = \M{U}_{x} \V{x}_0 + \M{U}_{u} \hatV{u},\\
\end{aligned}
\end{equation}
%
where
%
\begin{equation}
    \hatV{x} = (\V{x}_1, \dots, \V{x}_N),
    \quad
    \hatV{u} = (\V{u}_0, \dots, \V{u}_{N-1}).
\end{equation}
%
Matrices $\M{U}_{x}$ and $\M{U}_{u}$ are obtained by evaluating equation of the
dynamics recursively
%
\begin{equation}
\begin{aligned}
    \V{x}_{1} & = \M{A}_0 \V{x}_0 + \M{B}_0 \V{u}_0,\\
    \V{x}_{2} & = \M{A}_1 \V{x}_1 + \M{B}_1 \V{u}_1
                = \M{A}_1 \left( \M{A}_0 \V{x}_0 + \M{B}_0 \V{u}_0 \right) + \M{B}_1 \V{u}_1
                = \M{A}_1 \M{A}_0 \V{x}_0 +
                    \begin{bmatrix}
                        \M{A}_1 \M{B}_0 & \M{B}_1
                    \end{bmatrix}
                    \begin{bmatrix}
                        \V{u}_0 \\
                        \V{u}_1
                    \end{bmatrix}
                    ,
                    \\
    \dots     &,
\end{aligned}
\end{equation}
%
which results in
%
\begin{equation}
    \M{U}_{x} =
        \begin{bmatrix}
        \M{A}_0    \\
        \M{A}_1 \M{A}_0  \\
        \vdots           \\
        \M{A}_{N-1} \dots \M{A}_0 \\
        \end{bmatrix}
        ,
    \quad
    \M{U}_{u} =
        \begin{bmatrix}
        \M{B}_0                             & \M{0}                                 & \dots & \M{0} \\
        \M{A}_1 \M{B}_0                     & \M{B}_1                               & \dots & \M{0} \\
        \vdots                              & \vdots                                & \ddots& \vdots \\
        \M{A}_{N-1} \dots \M{A}_1 \M{B}_0   & \M{A}_{N-1} \dots \M{A}_2 \M{B}_1     & \dots & \M{B}_{N-1} \\
        \end{bmatrix}
        .
\end{equation}
%


In the case of time invariant system, matrices take a simpler form
%
\begin{equation}
    \M{U}_{x} =
        \begin{bmatrix}
        \M{A}    \\
        \M{A}^2  \\
        \vdots   \\
        \M{A}^N  \\
        \end{bmatrix}
        ,
    \quad
    \M{U}_{u} =
        \begin{bmatrix}
        \M{B}                   & \M{0}                 & \dots & \M{0} \\
        \M{A} \M{B}             & \M{B}                 & \dots & \M{0} \\
        \vdots                  & \vdots                & \ddots& \vdots \\
        \M{A}^{N-1} \M{B}       & \M{A}^{N-2} \M{B}     & \dots & \M{B} \\
        \end{bmatrix}
        ,
\end{equation}
%

After condensing, all state $\hatV{x} = (\V{x}_1, \dots, \V{x}_N)$ variables
are expressed through the initial state $\V{x}_0$ and control variables
$\hatV{u} = (\V{u}_0, \dots, \V{u}_{N-1})$. Therefore, $\hatV{x}$ and can be
eliminated from constraints and objectives of the \ac{MPC} problem.

%-------------------------------------------------------------------------------
\chapter{Trajectory of a foot in the air}
\label{app.swing_foot}
\acresetall
%-------------------------------------------------------------------------------

Trajectory of a foot in the air (\tn{swing foot}) is generated using cubic
polynomial of the form
%
\begin{equation}
    a_{\alpha} t^3 + b_{\alpha} t^2 + c_{\alpha}t + d_{\alpha} = s^{\alpha}_t,
\end{equation}
%
where $\alpha \in \{x,y,z\}$ denotes coordinate axis, $t$ is time instant,
$a_{\alpha},b_{\alpha},c_{\alpha},d_{\alpha}$ are coefficients, and
$s^{\alpha}_t$ is position of the swing foot at time $t$. Derivatives of the
cubic polynomial are
%
\begin{equation}\label{eq.swtraj_derivatives}
\begin{split}
    & 3a_{\alpha} t^2 + 2b_{\alpha} t + c_{\alpha} = \dot{s}^{\alpha}_t,\\
    & 6a_{\alpha} t + 2b_{\alpha} = \ddot{s}^{\alpha}_t,\\
    & 6a_{\alpha} = \dddot{s}^{\alpha}_t.\\
\end{split}
\end{equation}
%
There are four boundary conditions for the polynomial, the first two are
defined for the current swing foot state at $t = 0$:
%
\begin{equation}\label{eq.swtraj_bc1}
\begin{split}
    & d_{\alpha} = s^{\alpha}_0, \\
    & c_{\alpha} = \dot{s}^{\alpha}_0, \\
\end{split}
\end{equation}
%
where $s^{\alpha}_0$ and $\dot{s}^{\alpha}_0$ are initial position and
velocity. The remaining two conditions for landing time instant $t = l$ are:
%
\begin{equation}\label{eq.swtraj_bc2}
\begin{split}
    & a_{\alpha} l^3 + b_{\alpha} l^2 + \dot{s}^{\alpha}_0 l + s^{\alpha}_0 = s^{\alpha}_l,\\
    & 3a_{\alpha} l^2 + 2b_{\alpha} l + \dot{s}^{\alpha}_0 = \dot{s}^{\alpha}_l.
\end{split}
\end{equation}
%


Final position $s^{z}_l$ for trajectory along the $z$ axis is set to the step
height during the first half of the step and to $0$ during the second half:
%
\begin{equation}
    s^{z}_l =
    \left\{
        \begin{array}{ll}
            h   & t \le \frac{T_{step}}{2}; \\
            0   & t > \frac{T_{step}}{2}. \\
        \end{array}
    \right.
\end{equation}
%
The final $x,y$ positions are set to the next landing position (see the
\nameref{model.PPMZ} model)
%
\begin{equation}
    \begin{bmatrix}
        s^{x}_l\\
        s^{y}_l\\
    \end{bmatrix}
    =
    \hat{\contact}_{1}
    =
    \hat{\contact}_{0} + \hatM{B}_0\Delta\hat{\contact}_{0}.
\end{equation}
%
Velocity at the end of trajectory is set to zero $\dot{s}^{\alpha}_l = 0$.


%%%%%%%%%%%%%%%%%%%%%%%%%%%%%%%%%%%%%%%%%%%%%%%%%%%%%%%%%%%%%%%%%%%%%%%%%%%%%%%%%%%%%%%%%%%%%%%%%%%%%%
\section{Computation of the foot acceleration}

Whole body control requires current acceleration (at time $0$) of the swing
foot, which can be found as
%
\begin{equation}\label{eq.swtraj_acc1}
    2b_{\alpha} = \ddot{s}^{\alpha}_0.
\end{equation}
%
Hence it is necessary to find coefficients $b_{\alpha}$ from
equations~\cref{eq.swtraj_bc1} and~\cref{eq.swtraj_bc2}.


Trivial algebraic operations on \cref{eq.swtraj_bc2} and \cref{eq.swtraj_acc1}
lead to the following equation:
%
\begin{equation}\label{eq.swtraj_acc2}
    \ddot{s}^{\alpha}_0
    =
    \frac{6(s^{\alpha}_l - s^{\alpha}_0)}{l^2} - \frac{4\dot{s}^{\alpha}_0}{l}
    =
    \frac{6}{l^2} s^{\alpha}_l
    -
    \frac{6}{l^2} s^{\alpha}_0
    -
    \frac{4}{l} \dot{s}^{\alpha}_0.
\end{equation}
%
Consequently,
%
\begin{equation}
    \ddotV{s}_{0}
    =
    \begin{bmatrix}
        \ddotV{s}_0^{xy}\\
        \ddot{s}_0^z\\
    \end{bmatrix}
    =
    \underbrace{
        \frac{6}{l^2}
        \begin{bmatrix}
            \hatM{B}_0 \\
            0
        \end{bmatrix}
    }_{\objA_{\MT{sa}}}
    \Delta\hat{\contact}_{0}
    +
    \underbrace{
        \frac{6}{l^2}
        \begin{bmatrix}
            \hat{\contact}_{0} \\
            {s}_l^z
        \end{bmatrix}
        -
        \frac{6}{l^2}
        \V{s}_0
        -
        \frac{4}{l}
        \dotV{s}_0
    }_{\V{\objb}_{\MT{sa}}},
\end{equation}
%


%%%%%%%%%%%%%%%%%%%%%%%%%%%%%%%%%%%%%%%%%%%%%%%%%%%%%%%%%%%%%%%%%%%%%%%%%%%%%%%%%%%%%%%%%%%%%%%%%%%%%%
\section{Computation of the foot jerk}

The footstep adjustments are allowed during the whole duration of a step, which
may result in violent position changes near the end of a step. A
straightforward workaround is to block footstep adaptation when the foot
approaches the ground. Instead, we have chosen to minimize the $x,y$ components
of the foot jerk.


Using \cref{eq.swtraj_bc2} and \cref{eq.swtraj_derivatives} we find the jerk as
%
\begin{equation}\label{eq.swtraj_jerk}
    \dddot{s}^{\alpha}
    =
    -
    \frac{12}{l^3} s^{\alpha}_l
    +
    \frac{12}{l^3} s^{\alpha}_0
    +
    \frac{6}{l} \dot{s}^{\alpha}_0.
\end{equation}
%
Therefore, the $x,y$ components of the foot jerk are expressed as
%
\begin{equation}
    \dddotV{s}^{xy}
    =
    \underbrace{
        -
        \frac{12}{l^3}
        \hatM{B}_0
    }_{\objA_{\MT{sj}}}
    \Delta\hat{\contact}_{0}
    +
    \underbrace{
        -
        \frac{12}{l^3}
        \hat{\contact}_{0}
        +
        \frac{12}{l^3}
        s^{xy}_0
        +
        \frac{6}{l^2}
        \dot{s}^{xy}_0
    }_{\V{\objb}_{\MT{sj}}}
    .
\end{equation}

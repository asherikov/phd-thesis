\pdfbookmark[0]{Title}{title}
%\maketitle

%----------------------------------------------------------------------
% Title information
%----------------------------------------------------------------------
\Specialite{Automatique-Productique}
\Arrete{7 ao\^ut 2006}
\Auteur{Alexander Sherikov}
\Directeur{Bernard Brogliato}
%\CoDirecteur{Pierre-Brice Wieber} % optionnel
\CoEncadrant{Pierre-Brice Wieber}
\Laboratoire{de l'INRIA}
\EcoleDoctorale{l'\'Ecole Doctorale EEATS}
%\Titre{Préservation de l'équilibre et priorisation des tâches dans la commande du mouvement corps entier de robots humanoïdes}
\Titre{Balance preservation\\ and task prioritization\\ in whole body motion control\\ of humanoid robots}
%\Soustitre{Et le sous-titre}
%\Depot{\today}
\Depot{23 Mai 2016}

\Jury{

%    - Pascal Morin, Chargé de Recherche INRIA (CR) en détachement à l’UPMC,
%      Habilité à Diriger des Recherches (HDR)
%    - Olivier Stasse, Directeur de Recherche CNRS (DR) au LAAS
%    - Abderrahmane Kheddar, Directeur de Recherche CNRS (DR) au LIRMM
%    - Bernard BROGLIATO, Directeur de Recherche INRIA
%    - Pierre-Brice Wieber, Chargé de Recherche INRIA


%\UGTPresident{Mr \CM}{Doctorant, LIG}
%\UGTPresidente{Mme \CM}{Doctorante, LIG}
%\UGTExaminatrice{Mme \CM}{Doctorante, LIG}

\UGTRapporteur{Olivier Stasse}{Directeur de Recherche CNRS au LAAS}
\UGTRapporteur{Pascal Morin}{Chargé de Recherche INRIA en détachement à l’UPMC}

\UGTPresident{Abderrahmane Kheddar}{Directeur de Recherche CNRS au LIRMM}
%\UGTExaminateur{Abderrahmane Kheddar}{Directeur de Recherche CNRS au LIRMM}

\UGTDirecteur{Bernard Brogliato}{Directeur de Recherche INRIA}
%\UGTCoDirecteur{Pierre-Brice Wieber}{Charg\'e de Recherche INRIA}
\UGTCoEncandrant{Pierre-Brice Wieber}{Charg\'e de Recherche INRIA}
}

\frontmatter

%\maketitle %NE PAS LE METTRE
\MakeUGthesePDG
%----------------------------------------------------------------------

%\listoftodos
%\td{Delete TODO list}


%%%%%%%%%%%%%%%%%%%%%%%%%%%%%%%%%%%%%%%%%%%%%%%%%%%%%%%%%%%%%%%%%%%%%%%%%%%%%%%%
%-------------------------------------------------------------------------------
\cleardoublepage
\phantomsection
\pdfbookmark[0]{Abstract}{abstract}
\chapter*{Abstract}
%-------------------------------------------------------------------------------

One of the greatest challenges in robot control is closing the gap between the
motion capabilities of humans and humanoid robots. The difficulty lies in the
complexity of the dynamical systems representing the said robots: their
nonlinearity, underactuation, discrete behavior due to collisions and friction,
high number of degrees of freedom. Moreover, humanoid robots are supposed to
operate in non-deterministic environments, which require advanced real time
control. The currently prevailing approach to coping with these difficulties is
to impose various limitations on the motions and employ approximate models of
the robots. In this thesis, we follow the same line of research and propose a
new approach to the design of balance preserving whole body motion controllers.
The key idea is to leverage the advantages of whole body and approximate models
by mixing them within a single predictive control problem with strictly
prioritized objectives.


Balance preservation is one of the primary concerns in the control of humanoid
robots. Previous research has already established that anticipation of motions
is crucial for this purpose. We advocate that anticipation is helpful in this
sense as a way to maintain capturability of the motion, \IE, the ability to
stop. We stress that capturability of anticipated motions can be enforced with
appropriate constraints. In practice, it is common to anticipate motions using
approximate models in order to reduce computational effort, hence, a separate
whole body motion controller is needed for tracking. Instead, we propose to
introduce anticipation with an approximate model into the whole body motion
controller. As a result, the generated whole body motions respect the
capturability constraints and the anticipated motions of an approximate model
take into account whole body constraints and tasks. We pose our whole body
motion controllers as optimization problems with strictly prioritized
objectives. Though such prioritization is common in the literature, we believe
that it is often not properly exploited. We, therefore, propose several
examples of controllers, where prioritization is useful and necessary to
achieve desired behaviors. We evaluate our controllers in two simulated
scenarios, where a whole body task influences walking motions of the robot and
the robot optionally exploits a hand contact to maintain balance while
standing.


%\cleardoublepage
\clearpage
\phantomsection
\pdfbookmark[0]{Resume}{resume}
\chapter*{Résumé}

Un des plus grands défis dans la commande des robots est de combler l'écart
entre la capacité de mouvement de l'humain et des robots humanoïdes. La
difficulté réside dans la complexité des systèmes dynamiques représentant les
robots humanoïdes: la nonlinéarité, le sous-actionnement, le comportement
non-lisse en raison de collisions et de frottement, le nombre élevé de degrés
de liberté. De plus, les robots humanoïdes sont censés opérer dans des
environnements non-déterministes, qui exigent une commande temps réel avancée.
L'approche qui prévaut actuellement pour faire face à ces difficultés est
d'imposer diverses restrictions sur les mouvements et d'employer des modèles
approximatifs des robots. Dans cette thèse, nous suivons la même ligne de
recherche et proposons une nouvelle approche pour la conception de contrôleurs
corps entier qui preservent l'équilibre. L'idée principale est de tirer parti
des avantages des modèles approximatifs et de corps entier en les mélangeant
dans un seul problème de contrôle prédictif avec des objectifs strictement
hiérarchisés.


La préservation de l'équilibre est l'une des principales préoccupations dans la
commande des robots humanoïdes. Des recherches antérieures ont déjà établi que
l'anticipation des mouvements est essentiel à cet effet. Nous préconisons que
l'anticipation est utile dans ce sens comme un moyen de maintenir la
capturabilité du mouvement, \IE, la capacité de s'arrêter. Nous soulignons que
capturabilité des mouvements prévus peut être imposée avec des contraintes
appropriées. Dans la pratique, il est fréquent d'anticiper les mouvements du
robot à l'aide de modèles approximatifs afin de réduire l'effort de calcul, par
conséquent, un contrôleur séparé de mouvement du corps entier est nécessaire
pour le suivi. Au lieu de cela, nous proposons d'introduire l'anticipation avec
un modèle approximatif directement dans le contrôleur corps entier. En
conséquence, les mouvements du corps entier générés respectent les contraintes
de capturabilité et les mouvements anticipes du modèle approximatif prennent en
compte les contraintes et les tâches désirées pour le corps entier. Nous posons
nos contrôleurs du mouvement du corps entier comme des problèmes d'optimisation
avec des objectifs strictement hiérarchisés. Bien que cet ordre de priorité
soit commun dans la littérature, nous croyons qu'il est souvent mal exploité.
Par conséquent, nous proposons plusieurs exemples de contrôleurs, où la
hiérarchisation est utile et nécessaire pour atteindre les comportements
souhaités. Nous évaluons nos contrôleurs dans deux scénarios simulés, où la
tâche du corps entier du robot influence la marche et le robot exploite
éventuellement un contact avec la main pour maintenir son équilibre en étant
debout.


%%%%%%%%%%%%%%%%%%%%%%%%%%%%%%%%%%%%%%%%%%%%%%%%%%%%%%%%%%%%%%%%%%%%%%%%%%%%%%%%
%-------------------------------------------------------------------------------
%\cleardoublepage
\clearpage
\phantomsection
\pdfbookmark[0]{Acknowledgements}{acknowledgements}
\chapter*{Acknowledgements}
%-------------------------------------------------------------------------------

First and foremost, I would like to express my highest gratitude to
Pierre-Brice Wieber and Dimitar Dimitrov, who guided me through the past years
of my study. Their insights and advices were invaluable for my work. I would
also like to thank Bernard Brogliato and jury members Abderrahmane Kheddar,
Olivier Stasse, and Pascal Morin for their time and interest in this work.


Some of my research was performed in collaboration with Camille Brasseur and
Nestor Bohorquez-Dorante from BIPOP team, and Joven Agravante from LIRMM. I
must thank them for their effort and feedback.


I would like to personally thank Alejandro Blumentals and Jan Michalczyk, as
well as other present and past members of BIPOP team, for their general
assistance and support. I am also grateful to Diane Courtiol and Myriam Etienne
from INRIA for their administrative assistance during my stay in France.


%!sort -f
As an active user of free software during most of my professional life, I
cannot forget the people who invest their time and effort in development of
software tools and packages, which I employed during my work on this thesis:
\sn{Asymptote},
\sn{FreeBSD},
\sn{git},
\sn{Maxima},
\sn{Octave},
\sn{qpOASES},
\sn{TeX Live},
\sn{Vim},
and many others. Thank you all!


%%%%%%%%%%%%%%%%%%%%%%%%%%%%%%%%%%%%%%%%%%%%%%%%%%%%%%%%%%%%%%%%%%%%%%%%%%%%%%%%
%-------------------------------------------------------------------------------
%\cleardoublepage
\clearpage
\phantomsection
\pdfbookmark[0]{Contents}{contents}
\setcounter{tocdepth}{2}
\tableofcontents

%\cleardoublepage
\clearpage
\phantomsection
\pdfbookmark[1]{List of Figures}{listoffigures}
\listoffigures

%\cleardoublepage
%\phantomsection
%\pdfbookmark[1]{List of Tables}{listoftables}
%\listoftables
%\lstlistoflistings

%\cleardoublepage
%\phantomsection
%\pdfbookmark[1]{List of Algorithms}{listofalgorithms}
%\listofalgorithms

%\cleardoublepage
\clearpage
\phantomsection
\pdfbookmark[1]{List of Acronyms}{listofacronyms}

\acsetup{list-short-format=\bf,list-heading=chapter*,list-name=List of Acronyms}

\printacronyms
\acsetup{long-format=\tn}

%-------------------------------------------------------------------------------
\chapter{Introduction}
\label{ch.intro}
\acresetall
%-------------------------------------------------------------------------------

%%%%%%%%%%%%%%%%%%%%%%%%%%%%%%%%%%%%%%%%%%%%%%%%%%%%%%%%%%%%%%%%%%%%%%%%%%%%%%%%
%%%%%%%%%%%%%%%%%%%%%%%%%%%%%%%%%%%%%%%%%%%%%%%%%%%%%%%%%%%%%%%%%%%%%%%%%%%%%%%%
%%%%%%%%%%%%%%%%%%%%%%%%%%%%%%%%%%%%%%%%%%%%%%%%%%%%%%%%%%%%%%%%%%%%%%%%%%%%%%%%
\section{Context and motivation}

Humanoid robotics is undoubtedly one of the most fascinating areas in robotics
research. The reason for this is that the creation of robots, which are
comparable to humans at least in their motion capabilities, is expected to have
a tremendous social and industrial impact \cite{Kemp2008handbook}. At the
present moment, we are, however, quite far from this ultimate goal, even though
some remarkable progress has been made in the recent years \cite{DRCsite}.
While we hope that, eventually, humanoid robots will perform dangerous and
tedious tasks in our place, some valuable gains from studying these robots can
be obtained in short term. First of all, humanoid robots are very versatile and
a challenging research and educational platform, since they have to perform a
large variety of activities, such as sensing, manipulation, locomotion,
interaction with the environment, and others. Furthermore, these robots can be
employed in entertainment or play a role as an interactive interface for humans
\cite{Kemp2008handbook}. Control algorithms developed for humanoid robots may
find applications in prosthetic devices and exoskeletons, or in the generation
of natural and functional motions of animated characters
\cite{VanDePanne1997cgf}.


In the present thesis, we limit our focus to the motion control of humanoid
robots. This is a difficult problem due to the intrinsic complexity of the
dynamical systems representing the said robots: their nonlinearity,
underactuation, discrete behavior due to collisions and friction, high number
of degrees of freedom. Even the modeling and simulation of these systems is a
challenging problem on its own. Moreover, humanoid robots are supposed to
operate in non-deterministic environments, which require advanced real time
control. This real time control must often be realized using relatively weak
on-board computers, since a network connection to a more powerful computer is
not available or unreliable. Hence, there is a need for the development of
controllers, which adequately trade off computational complexity with quality
and generality of realizable motions. This is usually achieved by imposing
various limitations on the motions and employing approximate models of the
robots. Approximate models lack expressiveness of whole body models, but their
utilization results in smaller demands of computational resources.


The complexity of the kinematic structure of humanoid robots makes them
redundant with respect to typical motion objectives, \EG, the robot can perform
different tasks with its hands simultaneously. Consequently, it is common to
employ prioritization of these objectives in the controllers in order to
exploit this redundancy \cite{Kanoun2009thesis, Saab2013tro, Sentis2007thesis}.
Sometimes, however, prioritization is not well justified and is not meaningful.


The first goal of the present work is to demonstrate that the advantages of
whole body and an approximate model can be leveraged by mixing them within a
single predictive control problem. The second goal is to design controllers, in
which prioritization of objectives is useful and necessary to achieve new
behaviors.



%%%%%%%%%%%%%%%%%%%%%%%%%%%%%%%%%%%%%%%%%%%%%%%%%%%%%%%%%%%%%%%%%%%%%%%%%%%%%%%%
%%%%%%%%%%%%%%%%%%%%%%%%%%%%%%%%%%%%%%%%%%%%%%%%%%%%%%%%%%%%%%%%%%%%%%%%%%%%%%%%
%%%%%%%%%%%%%%%%%%%%%%%%%%%%%%%%%%%%%%%%%%%%%%%%%%%%%%%%%%%%%%%%%%%%%%%%%%%%%%%%
\section{Contribution}

\begin{itemize}
    \item One of the primary contributions of the present work is the
        introduction of a novel approach to the design of whole body motion
        controllers for humanoid robots. The key idea of the proposed approach
        is to employ the whole body model and an approximate model of the robot
        simultaneously within a single controller \cite{Sherikov2014humanoids,
        Sherikov2015humanoids}. The role of the whole body model is to allow
        instantaneous whole body motion control, while the approximate model is
        used for anticipation to ensure long term balance. Such a mix of models
        enables interplay between them, in particular, instantaneous whole body
        tasks can influence motions anticipated with an approximate model.

        The resulting controller can be perceived as a whole body \ac{MPC}
        problem, where the whole body model is replaced by an approximate model
        everywhere except the current time instant. Hence, in comparison with
        the whole body \ac{MPC}, our controllers sacrifice quality of the
        prediction for computational performance.


    \item The second contribution consists in several practical applications of
        strict prioritization of the objectives in the proposed whole body
        motion controllers and general \ac{MPC}:
        %
        \begin{itemize}
            \item We propose to employ prioritization of state constraints in
                an \ac{MPC} to implement a time optimal controller for an
                industrial manipulator \cite{alHomsi2016icra}.

            \item We introduce prioritization in contact force distribution in
                order to avoid the application of an optional hand contact
                force unless it is necessary for the execution of a whole body
                task or the preservation of balance
                \cite{Sherikov2015humanoids}.
        \end{itemize}
        %


    \item Finally, there is a number of technical contributions:
        %
        \begin{itemize}
            \item We implement coupling between the whole body model and two
                approximate models within a single controller and consider
                possible difficulties in this coupling
                \cite{Sherikov2014humanoids, Sherikov2015humanoids}.

            \item We discuss the construction of capturability constraints,
                which ensure that the robot can be stop without a fall, for
                several approximate models.

            \item Our whole body motion controllers with prioritized objectives
                serve as a testbed for a general purpose solver of \acf{PLLS}
                problems described in \cite{Dimitrov2015preprint}.

            \item We formulate the whole body motion controllers so that they
                are solved by the said solver in the most computationally
                efficient way.

            \item We evaluate the proposed controllers in simulations and
                develop the necessary software for performing these
                simulations.

            \item We contribute to the development of \ac{MPC} schemes for
                carrying objects in collaboration with humans
                \cite{Agravante2016icra} and for walking with varying \ac{CoM}
                height \cite{Brasseur2015humanoids}.
        \end{itemize}
        %
\end{itemize}


%%%%%%%%%%%%%%%%%%%%%%%%%%%%%%%%%%%%%%%%%%%%%%%%%%%%%%%%%%%%%%%%%%%%%%%%%%%%%%%%
\subsection{List of publications}

The work on this thesis resulted in the following publications in peer-reviewed
conferences
%
\begin{itemize}
    \item \bibentry{Sherikov2014humanoids}
    \item \bibentry{Sherikov2015humanoids}
    \item \bibentry{Brasseur2015humanoids}
    \item \bibentry{Agravante2016icra}
    \item \bibentry{alHomsi2016icra}
\end{itemize}
%

The author also contributed to two journal papers, which are currently under
review
%
\begin{itemize}
    \item \bibentry{Dimitrov2015preprint}
    \item \bibentry{Agravante2016preprint}
\end{itemize}
%


%%%%%%%%%%%%%%%%%%%%%%%%%%%%%%%%%%%%%%%%%%%%%%%%%%%%%%%%%%%%%%%%%%%%%%%%%%%%%%%%
%%%%%%%%%%%%%%%%%%%%%%%%%%%%%%%%%%%%%%%%%%%%%%%%%%%%%%%%%%%%%%%%%%%%%%%%%%%%%%%%
%%%%%%%%%%%%%%%%%%%%%%%%%%%%%%%%%%%%%%%%%%%%%%%%%%%%%%%%%%%%%%%%%%%%%%%%%%%%%%%%
\section{Outline}
% \crefrange{ch.balance}{ch.simulations}
This dissertation is composed of $5$ main Chapters
\ref{ch.balance}--\ref{ch.simulations}, a conclusive \cref{ch.conclusion} and
appendices. We start with discussion of our general point of view on balance
preservation of humanoid robots in \cref{ch.balance}. We declare anticipation
and capturability, \IE, the ability to stop, to be the basic concepts, which we
employ for the construction of balance preserving controllers. The next, rather
technical, \cref{ch.modeling} is devoted to the presentation of whole body and
approximate models employed in this work. In \cref{ch.mpc}, we move on to
anticipation using approximate models. We also consider capturability enforcing
constraints for various approximate models, and introduce the idea of mixing
whole body and approximate models. \cref{ch.optimization} is focused on the
\acf{PLLS} optimization framework, which is used to define our controllers. We
outline the idea of prioritization of objectives and give several examples of
its applications. Finally, we discuss techniques, which allow for efficient
solution of \ac{PLLS} problems. In \cref{ch.simulations}, we consider two whole
body motion controllers presented in \cite{Sherikov2014humanoids,
Sherikov2015humanoids} in detail and discuss their performance in simulations.
In addition to that, we overview results of several works performed in
collaboration.


%%%%%%%%%%%%%%%%%%%%%%%%%%%%%%%%%%%%%%%%%%%%%%%%%%%%%%%%%%%%%%%%%%%%%%%%%%%%%%%%
%%%%%%%%%%%%%%%%%%%%%%%%%%%%%%%%%%%%%%%%%%%%%%%%%%%%%%%%%%%%%%%%%%%%%%%%%%%%%%%%
%%%%%%%%%%%%%%%%%%%%%%%%%%%%%%%%%%%%%%%%%%%%%%%%%%%%%%%%%%%%%%%%%%%%%%%%%%%%%%%%
\section{Notation}\label{sec.notation}

\begin{description}
    \item[Software names] \hfill \\
        Names of programs and software libraries, names of constants, variables
        and functions that are used in programs are typed in a monospaced font:
        \sn{Eigen}, \sn{LexLS}.

    \item[General scalars, vectors, matrices] \hfill
        \begin{itemize}[leftmargin=0cm]
            \item Vectors and matrices are denoted by letters in a bold font:
                $\V{v}$, $\M{M}$, $\M{\mathcal{A}}$.

            \item Scalars are denoted using the standard italic of calligraphic
                font: $N, n, \mathcal{K}$.

            \item $\T{(\cdot)}$ -- transpose of a matrix or a vector.

            \item $\CROSS[(\cdot)]$ -- a skew-symmetric matrix used for
                representation of a cross product of two three dimensional
                vectors as a product of a matrix and a vector:
                %
                \begin{equation}
                    \V{v}
                    =
                    \begin{bmatrix}
                        v^x\\
                        v^y\\
                        v^z
                    \end{bmatrix},
                    \quad
                    \CROSS[\V{v}]
                    =
                    \begin{bmatrix}
                        0      &   -v^z   &   v^y\\
                        v^z    &   0      &   -v^x\\
                        -v^y   &   v^x    &   0 \\
                    \end{bmatrix}.
                \end{equation}

            \item Block diagonal matrices:
                %
                \begin{equation}
                    \begin{gathered}
                        \diag{2}{\M{M}} =
                        \begin{bmatrix}
                            \M{M}   &   \M{0}\\
                            \M{0}   &   \M{M}\\
                        \end{bmatrix}
                        ,
                        \quad
                        \diag{k = 1 \dots 2}{\M{M}_k} =
                        \begin{bmatrix}
                            \M{M}_1 &   \M{0}  \\
                            \M{0}   &   \M{M}_2\\
                        \end{bmatrix}
                        ,
                        \\
                        \diag{}{\M{M}, \M{R}} =
                        \begin{bmatrix}
                            \M{M} &   \M{0}  \\
                            \M{0}   &   \M{R}\\
                        \end{bmatrix}
                        .
                    \end{gathered}
               \end{equation}

            \item Stacked vectors and matrices:
                %
                \begin{equation}
                    \V{v} = (\V{v}_1, \dots, \V{v}_n) = \begin{bmatrix} \V{v}_1 \\ \vdots \\ \V{v}_n \\ \end{bmatrix},
                    \quad
                    \M{M} = (\M{M}_1, \dots, \M{M}_n) = \begin{bmatrix} \M{M}_1 \\ \vdots \\ \M{M}_n \\ \end{bmatrix}.
                \end{equation}

            \item Inequalities between vectors $\V{v} \ge \V{r}$ are interpreted
                component-wise.
        \end{itemize}


    \item[Special matrices and vectors] \hfill
        \begin{itemize}[leftmargin=0cm]
            \item $\M{I}$ -- an identity matrix. $\M{I}_n$ -- $n \CROSS n$ identity matrix.

            \item $\M{I}_{(\cdot)}$ -- a selection matrix, examples:
                \begin{itemize}
                    \item $\Itorques$ -- a torque selection matrix;

                    \item $\M{I}_{\alpha}$, where $\alpha$ is a combination of
                        $\{x,y,z\}$, selects components of a 3d vector:
                        %
                        \begin{equation}
                            \Ix = \begin{bmatrix} 1 & 0 & 0 \end{bmatrix}, \quad
                            \Ixz = \begin{bmatrix} 1 & 0 & 0 \\ 0 & 0 &  1 \end{bmatrix}, \quad
                            \Izy = \begin{bmatrix} 0 & 0 & 1 \\ 0 & 1 &  0 \end{bmatrix}.
                        \end{equation}
                \end{itemize}

            \item $\M{0}$ -- a matrix of zeros. $\M{0}_{n,m}$ -- $n \CROSS m$ matrix of zeros.

            \item $\M{J} = (\M{J}_{\TRAN}, \M{J}_{\ROT})$ -- a Jacobian matrix
                with translational $\M{J}_{\TRAN}$ and rotational
                $\M{J}_{\ROT}$ parts.
        \end{itemize}


    \item[Reference frames] \hfill
        \begin{itemize}[leftmargin=0cm]
            \item Frames are denoted using a sans-serif font: $\FRAME{A}$. All
                considered frames are orthonormal.

            \item $\V[A]{v}$ -- vector expressed in frame $\FRAME{A}$.

            \item $\M[B][A]{R}$ -- rotation matrix from from frame $\FRAME{B}$
                to frame $\FRAME{A}$.

            \item The global frame is implicit and is not denoted by any
                letter, \EG, $\M[B][]{R}$ rotates from frame $\FRAME{B}$ to the
                global frame.
        \end{itemize}


    \item[Sets] \hfill
        \begin{itemize}[leftmargin=0cm]
            \item The sets are denoted using a blackboard bold font: $\SET{A}$.

            \item $\RR$ is the set of real numbers.

            \item $\RR_{\ge 0}, \RR_{> 0}$ are the sets of non-negative and positive real numbers.

            \item $\RR^n$ is the set of real-valued vectors.

            \item $\RR^{n \CROSS m}$ is the set of real-valued matrices.
        \end{itemize}


    \item[Other] \hfill
        \begin{itemize}[leftmargin=0cm]
            \item Function names in mathematical expressions are written in the
                regular font: $\FUNC{func}(\V{x}, \V{y})$.

            \item $\NORME{\cdot}$ denotes the Euclidean norm.
        \end{itemize}
\end{description}
